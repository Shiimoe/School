\documentclass{article}
\usepackage{amsmath}
\usepackage{amssymb}
\usepackage[utf8]{inputenc}
\usepackage{enumitem}
\setlength{\parskip}{1em}

\title{Maths Semester II}
\author{pcysf6 }
\date{January 2020}

\newcommand{\ODE}[3]{#1\frac{d^2y}{dx^2} + #2\frac{dy}{dx} + #3y}

\begin{document}

\maketitle

\section{First Order Differential Equations(ODEs)}
\subsection{Separable ODEs}
A separable ODE is one in which the functions of both x and y can be separated.
\begin{align*}
    \frac{dy}{dx} &= \frac{g(x)}{f(y)} \\
    \frac{dy}{dx} f(y) &= g(x)
\end{align*}
This can then be rearranged and integrated so that
$$\int f(y) dy = \int g(x) dx\\$$
\textrm{Example}
\begin{align*}
    \frac{dy}{dx} &= e^{x+y} = e^xe^y\\
    \int e^{-y} dy &= \int e^x dx\\
    -e^{-y} &= e^x + c\\
    e^x + e^{-y} &= c\\
\end{align*}
$c$ is just a constant so its sign does not matter at this point.\\
We can then solve explicitly for $y$. 
\begin{align*}
    e^{-y} &= c - e^x\\
    y &= -\ln(c-e^x)
\end{align*}
\newpage

\subsection{Boundary and Initial Conditions}
The general solution to a 1st Order DE always contains one undefined constant of integration, like $c$. 
A boundary (for x) or initial (for t) condition is given, typcially y(x) at a given x value.

Example:
T goes from 90$^\circ$c to 70$^\circ$c in 10 minutes(t), 
the room temperature is 20$^\circ$c(T$_0$), find T after 20 minutes.
\begin{align*}
    \frac{dT}{dt} &= -\alpha(T-T_0)\\
    \int \frac{dT}{T-T_0} &= -\alpha \int dt\\
    \ln(T-T_0) &= -\alpha t + c\\
    T-T_0 &= e^{-\alpha t + c} = Ae^{-\alpha t}\\
    A &= e^c
\end{align*}
We can now use our initial conditions to solve for T(20)
\begin{align*}
    \textrm{When } t &= 0\\
    T - T_0 &= 90 - 20 = 70\\
    A &= 70\\
    \textrm{When } t &= 10\\
    T &= T_0 + Ae^{-\alpha t}\\
    70 &= 20 + 70e^{-10\alpha }\\
    e^{-10\alpha} &= \frac{70-20}{70} = \frac{5}{7}\\
    \textrm{When } t &= 20\\
    T &= 20 + 70e^{-20\alpha}\\
    (e^{-10\alpha})^2 &= e^{-20\alpha}\\
    T &= 20 + 70\left(\frac{5}{7}\right)^2 = 55.7^\circ c
\end{align*}
\newpage
\subsection{Homoegeneous ODEs}
Replace $y$ with $yt$ and $x$ with $xt$, if all the $t$s cancel then the ODE is homogeneous.
\begin{align*}
    \frac{dy}{dx} &= f\left(\frac{y}{x}\right)\\
    \textrm{let }v &= \frac{y}{x} \textrm{ so } y = vx\\
    \frac{dy}{dx} &= \frac{d}{dx}vx\\
\end{align*}
$v$ is a function of $x$ so we must use the product rule.
\begin{align*}
    \frac{dy}{dx} &= v + x\frac{dv}{dx} = f(v)\\
    \frac{dv}{dx} &= \frac{f(v) - v}{x}\\
\end{align*}
This is now a separable function, and can be solved as before.

Example: $y(1) = 1$
\begin{align*}
    \frac{dy}{dx} &= \frac{y}{x^2}(x-y) = \frac{y}{x} - \left(\frac{y}{x}\right)^2\\
    \textrm{let }v &= \frac{y}{x} \textrm{ so } y = vx\\
    \frac{dy}{dx} &= v + x\frac{dv}{dx} = v - v^2 \quad\textrm{ separable}\\
    \int \frac{-dv}{v^2} &= \int \frac{dx}{x}\\
    \frac{1}{v} &= \ln x + c = \frac{x}{y}\\
    y &= \frac{x}{\ln x + c}\\
    y(1) &= 1\\
    1 &= \frac{1}{\ln(1) + c}\\
    c &= 1\\
\end{align*}
\newpage
\subsection{Linear ODEs}
Linear ODEs always take the form of $$\frac{dy}{dx} + y\cdot P(x) = Q(x)$$
When $Q(x) = 0$ this becomes a separable ODE and can be solved as before such that
\begin{align*}
    dy \cdot \frac{1}{y} &= -P(x) dx\\
    y &= Ae^{\int -P(x)dx}\quad\textrm{where } A = e^c
\end{align*}
When $P(x) = 0$ this becomes a simple separable ODE and can be solved by integration.

Recall the product rule $\frac{d}{dx}(u\cdot v) = u'v + uv'$\\
This is strikingly close to our Linear ODE, except we are missing a factor, the integration factor, $\mu(x)$.
We want $\mu$ to be such that $\mu' = P(x)\mu$, allow us to do some algebraic manipulation.
\begin{align*}
    \mu' &= P(x)\mu\\
    P(x) &= \frac{\mu'}{\mu} = \frac{d}{dx}\ln(\mu)\\
    \int P(x) dx &= \ln(\mu)\\
    \mu &= e^{\int P(x) dx}
\end{align*}
We can then multiply by our integrating factor, then apply the chain rule, however this is best shown with an example.
\begin{align*}
    \cos(x)\frac{dy}{dx} + \sin(x)y &= 1\\
    \frac{dy}{dx} + \tan(x)y &= \sec(x) \quad\textrm{Linear ODE}\\
    \mu &= e^{\int \tan(x) dx}\\
    \int \tan(x) dx &= \ln(sec(x))\\
    \mu &= \sec(x) \quad\textrm{Multiply ODE by $\mu$}\\
    \frac{dy}{dx}\sec(x) + \sec(x)\tan(x)y &= \sec^2(x)\\
\end{align*}
\newpage
Since $\frac{d}{dx}\sec(x) = \sec(x)\tan(x)$ we should now be able to see how the LHS of our equation can be withdrawn 
into its chain rule form.
\begin{align*}
    \frac{d}{dx}\sec(x)\cdot y &= \frac{dy}{dx}\sec(x) + \sec(x)\tan(x)y\\
    \frac{d}{dx}\sec(x)\cdot y &= sec^2(x)\\
    \sec(x) \cdot y &= \int \sec^2(x) = \tan(x) + c\\
    y &= \sin(x) + c\cdot \cos(x)
\end{align*}
\subsection{Exact ODEs}
When a variable in a multi variable function $f(x,y)$ has variables that rely on another e.g. 
$y = y(t), x = x(t)$ then the derivative of that function is known as the total derivative and
 is given as $$\frac{d}{dt}f(x,y) = \frac{\partial f}{\partial x} \frac{dx}{dt} + \frac{\partial f}{\partial y} \frac{dy}{dt}$$
 This can also occur for when one variable is the function of another e.g. $x = x, y = y(x)$, the total derivative of such is 
\begin{equation}
\label{eqn:Total Derivative}
\frac{d}{dx}f(x,y) = \frac{\partial f}{\partial x} + \frac{\partial f}{\partial y} \frac{dy}{dx}
\end{equation}
Exact ODEs take the form $$M(x,y) + N(x,y)\frac{dy}{dx}=0$$
Where M and N must be separated by a plus. 
As you can see, this is in the same form as a the derivative of the function from equation (1),
 thus we can assume that $$\frac{\partial f}{\partial x} = M \quad \frac{\partial f}{\partial y} = N$$
In order to verify this assumption we can test an 'exactness condition' 
\begin{align*}
    \frac{\partial^2f}{\partial x \partial y} &= \frac{\partial^2 f}{\partial y \partial x}\\
    \frac{\partial M}{\partial y} &= \frac{\partial N}{\partial x}
\end{align*}
We must now find a function to satisfy these conditions, however this is best left to an example.
\newpage
\begin{align*}
     x + y^2 + 2xy\frac{dy}{dx} &= 0\\
     M = x + y^2 & \quad N = 2xy\\
     \frac{\partial M}{\partial y} &= 2y\\
     \frac{\partial N}{\partial x} &= 2y\\
\end{align*}
This satisfies the 'exactness condition', and so we can begin to find the function that solves the ODE.
\begin{align*}
    M &= \frac{\partial f}{\partial x}\\
    \int M dx &= f = \frac{1}{2}x^2 + xy^2 + g(y)\\
    N &= \frac{\partial f}{\partial y}\\
    \int N dy &= f = xy^2 + h(x)\\
\end{align*}
For $\int M dx = \int N dy$ to be true their constants must satisfy each other, thus $g(y) = 0, h(x) = \frac{1}{2}x^2$
$$f(x,y) = \frac{1}{2}x^2 + xy^2 + c = 0$$

\newpage
\section{Second Order ODEs}

\subsection{Introduction}
They take the form of
\begin{align*}
    m(x)\frac{d^2y}{dx^2} + p(x)\frac{dy}{dx} + q(x)y &= r(x)\\
    r(x) &= 0 \quad\text{Homogeneous}\\
    r(x) &\not= 0 \quad\text{Inhomogeneous}\\
\end{align*}
The general solution is 
$$y(x) = Ay_1(x) + By_2(x)$$
Where $y_1$ \& $y_2$ are two functions of x and A \& B are integration constants.
We normally assume we have constant coefficients, as in driven oscillators.
$$m\frac{d^2y}{dx^2} + b\frac{dy}{dx} + ky = F_0\cos(\omega t)$$

\subsection{Homogeneous}
Consider the second order homogeneous ODE
$$\ODE{}{a}{b} = 0$$
We normally take an educated guess at a solution, e.g. $y=e^{mx}$, where m is some constant. Thus 
$$\frac{dy}{dx} = me^{mx} \quad\quad \frac{d^2y}{dx^2} = m^2e^{mx}$$
Subbing this back into our original homogeneous equation and then dividing by $e^{mx}$ gives us
\begin{align*}
    (m^2 + am + b) &= 0 \quad\text{axuillary equation}\\
    (m-m_1)(m-m_2) &= 0 \quad\text{$m_1$ \& $m_2$ are roots}\\
\end{align*}
There are three cases to consider, when $m_1$ \& $m_2$ are 
\begin{enumerate}[label=\alph*.]
    \item distinct and real 
    \item distance and complex
    \item the same
\end{enumerate}

\subsubsection{Case a.}
Where $m_1 \not= m_2$ and both $m_1$ \& $m_2$ are real. From our educated guess we know
$$y_1(x) = e^{m_1x} \quad\quad y_2(x) = e^{m_2x}$$
So the general solution is 
$$y(x) = Ae^{m_1x} + Be^{m_2x}$$
Example:
\begin{align*}
    \ODE{3}{-5}{-2} &= 0\\
    3m^2 - 5m - 2 &= 0 \quad\text{auxillary equation}\\
    (3m+1)(m-2) &= 0\\
    m_1 &= \frac{-1}{3}\\
    m_2 &= 2\\
    y(x) &= Ae^{-x/3} + Be^{2x} \quad\text{general solution}
\end{align*}

\subsubsection{Case b.}
Where $m_1 \not= m_2$ and both $m_1$ \& $m_2$ are complex. We know that
$$y_1(x) = e^{m_1x} \quad\quad y_2(x) = e^{m_2x}$$
But now $y_1$ \& $y_2$ are complex.
This means we can rewrite the general solution using Euler's Formula.
\begin{align*}
    \textrm{let}\quad m_1 &= \mu + i\lambda\\
    \textrm{let}\quad m_2 &= \mu - i\lambda
\end{align*}
So the general solution is
\begin{align*}
    y(x) = Ae^{m_1x} + Be^{m_2x} &= Ae^{(\mu + i\lambda)x} + Be^{(\mu - i\lambda)x}\\
    &= e^{\mu x}[Ae^{i\lambda x} + Be^{-i\lambda x}]\\
    &= e^{\mu x}[A(\cos(\lambda x) + i\sin(\lambda x)) + B(\cos(\lambda x) - i\sin(\lambda x))]\\
    &= e^{\mu x}[(A+B)(\cos(\lambda x) + (A-B)i\sin(\lambda x))\\
    &= e^{\mu x}[(C)(\cos(\lambda x) + (D)\sin(\lambda x))\\
\end{align*}
The final general solution is
\[e^{\mu x}[(C)(\cos(\lambda x) + (D)\sin(\lambda x))\]
Where $C = A+B$ \& $D=i(A-B)$ are both integration constants.

\subsubsection{Case c.}
Where $m_1 = m_2$. Thus
\[y_1 = e^{m_1x}\]
So we need another solution. We will prove this in a later section.
\begin{align*}
    y_2(x) &= xe^{m_1x}\\
    y(x) &= Ay_1(x) + By_2(x) \quad\text{general solution}\\
    y(x) &= (A+Bx)e^{m_1x} 
\end{align*}
Example:
\begin{align*}
    \ODE{}{2}{} &= 0\\
    m^2 + 2m + 1 &= (m+1)^2 = 0 \quad\text{auxillary equation}\\
    m &= -1\\
    y_1(x) &= e^{-x}\\
    y_2(x) &= xe^{-x}\\
    y(x) &= (A+Bx)e^{-x}\\ \quad\text{general solution}
\end{align*}
We can check that $y_2$ is a solution.
\begin{align*}
    \frac{dy_2}{dx} &= e^{-x}-xe^{-x} = (1-x)e^{-x}\\
    \frac{d^2y_2}{dx^2} &= -e^{-x}-e^{-x}+xe^{-x} = (x-2)e^{-x}\\
    \ODE{}{2}{} &= 0\\ \quad\text{original ODE}\\
    [(x-2) + 2(1-x) + x]e^{-x} &= 0\\
    0 &= 0
\end{align*}
\end{document}
