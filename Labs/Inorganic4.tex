\documentclass{article}
\usepackage[utf8]{inputenc}
\usepackage{enumerate}
\usepackage{enumitem}
\usepackage{mhchem}

\newcommand{\super}{\textsuperscript}
\newcommand{\numero}{N\super{\underline{o}}}

\newcommand\bfit[1]{\textbf{\textit{#1}}}

\newlist{questions}{enumerate}{2}
\setlist[questions,1]{align=left,label={\bfseries\large {Question} \numero\arabic*.}}
\setlist[questions,2]{align=left,label={(\alph*)}}
\newcommand\question[1]{{\large \item\ #1}}1

\newcommand\paren[1]{\left(#1\right)}

\renewcommand{\thesection}{\S\,\arabic{section}}
\renewcommand{\thesubsection}{\thesection{}.\,\arabic{subsection}}

\title{Synthesis of \ce{K_3[Fe(C_2O_4)_3]*H_2O}}
\author{Samuel J. Frost}
\date{8\textsuperscript{th} of February, 2020}

\begin{document}
\maketitle

\section{Reaction Theory}

\centerline{\large\ce{Fe^{2+} ->[C_2O_4^{2-}] FeC_2O_4*H_2O}}
\begin{center}
    \begin{tabular}{c|c|c}
        Reagent & Moles & Mass\\
        \hline
        iron(II) ammonium sulphate & 0.0127 mol & 5 g\\
        oxalic acid dihydrate & 0.0396 mol & 5 g\\
    \end{tabular}
\end{center}

\centerline{\large\ce{FeC_2O_4*H_2O ->[H_2O_2][K_2C_2O_4*H_2O] Fe(OH)_3 + K_3Fe(C_2O_4)_3}}
\begin{center}
    \begin{tabular}{c|c|c}
        Reagent & Moles & Mass or Volume\\
        \hline
        hydrogen peroxide & 0.0179 mol & 10 cm$^3$\\
        potassium oxalate monohydrate & 0.0271 mol & 3.5 g \\
    \end{tabular}
\end{center}

\newpage
\section{Experimental Method}
\subsection{[Fe(C$_2$O$_4$)]$\cdot$ 2H$_2$O}

Oxalic acid dihydrate (5 g) was dissolved in deionised water (50 cm$^3$). Iron(II) ammonium
sulphate hexahydrate (5 g) was dissolved in warm deionised water (20 cm$^3$) and then acidified
with dilute sulphuric acid (2 M, 1 cm$^3$). The mixture was stirred rapidly. The oxalic acid 
solution (25 cm$^3$) was added and heated to boiling. The mixture was allowed to settle and then
decanted, it was mixed with hot deionised water (15 cm$^3$) before being decanted again. The
product was collected via Büchner filtration and washed with hot deionised water followed by
acetone before being dried (2.40 g).

\subsection{\ce{K_3[Fe(C_2O_4)_3]*H_2O}}

The iron(II) oxalate from \S 2.1 was suspended in a warm solution of potassium oxalate monohydrate
(3.5 g) in deionised water (10cm$^3$). Hydrogen Peroxide was added dropwise (1.786 M, 10 cm$^3$).
The mixture was heated to boiling whilst slowly adding the oxalic acid solution (8 cm$^3$). A
further 3 cm$^3$ of the solution was added slowly. The solution was filtered through fluted
filter paper and methylated spirits (10 cm$^3$) were added. The solution was allowed to cool
and crysatalise before being collected through a Büchner filter and washed with a 1:1 methylated
spirits:deionised water solution and acetone. The final product was  dried in a vacuum desiccator
in the dark (1.80 g).

\newpage
\section{Interpretation}

\begin{questions}
    \question{What is the purpose of the H$_2$O$_2$ in the preparation of 
    K$_3$[Fe(C$_2$O$_4$)$_3$]$\cdot$ 3H$_2$O}

    The H$_2$O$_2$ oxidises the Fe(II) in the iron(II) oxalate to Fe(III), thus converting it
    to iron(III) oxalate.\\

    \question{In the (C$_2$O$_4$)$^{2-}$ anion are all four C-O interatomic distances equal or are
    two shorter than the other two?}

    As the oxalate anion undergoes resonance the bond lengths will be an average of both the
    C=O double bond and the C-O single bond, thus every bond is the same length.
\end{questions}

\end{document}